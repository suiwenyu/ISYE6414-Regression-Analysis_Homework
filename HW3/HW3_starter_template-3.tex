% Options for packages loaded elsewhere
\PassOptionsToPackage{unicode}{hyperref}
\PassOptionsToPackage{hyphens}{url}
%
\documentclass[
]{article}
\usepackage{amsmath,amssymb}
\usepackage{lmodern}
\usepackage{iftex}
\ifPDFTeX
  \usepackage[T1]{fontenc}
  \usepackage[utf8]{inputenc}
  \usepackage{textcomp} % provide euro and other symbols
\else % if luatex or xetex
  \usepackage{unicode-math}
  \defaultfontfeatures{Scale=MatchLowercase}
  \defaultfontfeatures[\rmfamily]{Ligatures=TeX,Scale=1}
\fi
% Use upquote if available, for straight quotes in verbatim environments
\IfFileExists{upquote.sty}{\usepackage{upquote}}{}
\IfFileExists{microtype.sty}{% use microtype if available
  \usepackage[]{microtype}
  \UseMicrotypeSet[protrusion]{basicmath} % disable protrusion for tt fonts
}{}
\makeatletter
\@ifundefined{KOMAClassName}{% if non-KOMA class
  \IfFileExists{parskip.sty}{%
    \usepackage{parskip}
  }{% else
    \setlength{\parindent}{0pt}
    \setlength{\parskip}{6pt plus 2pt minus 1pt}}
}{% if KOMA class
  \KOMAoptions{parskip=half}}
\makeatother
\usepackage{xcolor}
\usepackage[margin=1in]{geometry}
\usepackage{color}
\usepackage{fancyvrb}
\newcommand{\VerbBar}{|}
\newcommand{\VERB}{\Verb[commandchars=\\\{\}]}
\DefineVerbatimEnvironment{Highlighting}{Verbatim}{commandchars=\\\{\}}
% Add ',fontsize=\small' for more characters per line
\usepackage{framed}
\definecolor{shadecolor}{RGB}{248,248,248}
\newenvironment{Shaded}{\begin{snugshade}}{\end{snugshade}}
\newcommand{\AlertTok}[1]{\textcolor[rgb]{0.94,0.16,0.16}{#1}}
\newcommand{\AnnotationTok}[1]{\textcolor[rgb]{0.56,0.35,0.01}{\textbf{\textit{#1}}}}
\newcommand{\AttributeTok}[1]{\textcolor[rgb]{0.77,0.63,0.00}{#1}}
\newcommand{\BaseNTok}[1]{\textcolor[rgb]{0.00,0.00,0.81}{#1}}
\newcommand{\BuiltInTok}[1]{#1}
\newcommand{\CharTok}[1]{\textcolor[rgb]{0.31,0.60,0.02}{#1}}
\newcommand{\CommentTok}[1]{\textcolor[rgb]{0.56,0.35,0.01}{\textit{#1}}}
\newcommand{\CommentVarTok}[1]{\textcolor[rgb]{0.56,0.35,0.01}{\textbf{\textit{#1}}}}
\newcommand{\ConstantTok}[1]{\textcolor[rgb]{0.00,0.00,0.00}{#1}}
\newcommand{\ControlFlowTok}[1]{\textcolor[rgb]{0.13,0.29,0.53}{\textbf{#1}}}
\newcommand{\DataTypeTok}[1]{\textcolor[rgb]{0.13,0.29,0.53}{#1}}
\newcommand{\DecValTok}[1]{\textcolor[rgb]{0.00,0.00,0.81}{#1}}
\newcommand{\DocumentationTok}[1]{\textcolor[rgb]{0.56,0.35,0.01}{\textbf{\textit{#1}}}}
\newcommand{\ErrorTok}[1]{\textcolor[rgb]{0.64,0.00,0.00}{\textbf{#1}}}
\newcommand{\ExtensionTok}[1]{#1}
\newcommand{\FloatTok}[1]{\textcolor[rgb]{0.00,0.00,0.81}{#1}}
\newcommand{\FunctionTok}[1]{\textcolor[rgb]{0.00,0.00,0.00}{#1}}
\newcommand{\ImportTok}[1]{#1}
\newcommand{\InformationTok}[1]{\textcolor[rgb]{0.56,0.35,0.01}{\textbf{\textit{#1}}}}
\newcommand{\KeywordTok}[1]{\textcolor[rgb]{0.13,0.29,0.53}{\textbf{#1}}}
\newcommand{\NormalTok}[1]{#1}
\newcommand{\OperatorTok}[1]{\textcolor[rgb]{0.81,0.36,0.00}{\textbf{#1}}}
\newcommand{\OtherTok}[1]{\textcolor[rgb]{0.56,0.35,0.01}{#1}}
\newcommand{\PreprocessorTok}[1]{\textcolor[rgb]{0.56,0.35,0.01}{\textit{#1}}}
\newcommand{\RegionMarkerTok}[1]{#1}
\newcommand{\SpecialCharTok}[1]{\textcolor[rgb]{0.00,0.00,0.00}{#1}}
\newcommand{\SpecialStringTok}[1]{\textcolor[rgb]{0.31,0.60,0.02}{#1}}
\newcommand{\StringTok}[1]{\textcolor[rgb]{0.31,0.60,0.02}{#1}}
\newcommand{\VariableTok}[1]{\textcolor[rgb]{0.00,0.00,0.00}{#1}}
\newcommand{\VerbatimStringTok}[1]{\textcolor[rgb]{0.31,0.60,0.02}{#1}}
\newcommand{\WarningTok}[1]{\textcolor[rgb]{0.56,0.35,0.01}{\textbf{\textit{#1}}}}
\usepackage{graphicx}
\makeatletter
\def\maxwidth{\ifdim\Gin@nat@width>\linewidth\linewidth\else\Gin@nat@width\fi}
\def\maxheight{\ifdim\Gin@nat@height>\textheight\textheight\else\Gin@nat@height\fi}
\makeatother
% Scale images if necessary, so that they will not overflow the page
% margins by default, and it is still possible to overwrite the defaults
% using explicit options in \includegraphics[width, height, ...]{}
\setkeys{Gin}{width=\maxwidth,height=\maxheight,keepaspectratio}
% Set default figure placement to htbp
\makeatletter
\def\fps@figure{htbp}
\makeatother
\setlength{\emergencystretch}{3em} % prevent overfull lines
\providecommand{\tightlist}{%
  \setlength{\itemsep}{0pt}\setlength{\parskip}{0pt}}
\setcounter{secnumdepth}{-\maxdimen} % remove section numbering
\ifLuaTeX
  \usepackage{selnolig}  % disable illegal ligatures
\fi
\IfFileExists{bookmark.sty}{\usepackage{bookmark}}{\usepackage{hyperref}}
\IfFileExists{xurl.sty}{\usepackage{xurl}}{} % add URL line breaks if available
\urlstyle{same} % disable monospaced font for URLs
\hypersetup{
  pdftitle={HW3 Peer Assessment},
  hidelinks,
  pdfcreator={LaTeX via pandoc}}

\title{HW3 Peer Assessment}
\author{}
\date{\vspace{-2.5em}}

\begin{document}
\maketitle

\hypertarget{background}{%
\section{Background}\label{background}}

The fishing industry uses numerous measurements to describe a specific
fish. Our goal is to predict the weight of a fish based on a number of
these measurements and determine if any of these measurements are
insignificant in determining the weigh of a product. See below for the
description of these measurments.

\hypertarget{data-description}{%
\subsection{Data Description}\label{data-description}}

The data consists of the following variables:

\begin{enumerate}
\def\labelenumi{\arabic{enumi}.}
\tightlist
\item
  \textbf{Weight}: weight of fish in g (numerical)
\item
  \textbf{Species}: species name of fish (categorical)
\item
  \textbf{Body.Height}: height of body of fish in cm (numerical)
\item
  \textbf{Total.Length}: length of fish from mouth to tail in cm
  (numerical)
\item
  \textbf{Diagonal.Length}: length of diagonal of main body of fish in
  cm (numerical)
\item
  \textbf{Height}: height of head of fish in cm (numerical)
\item
  \textbf{Width}: width of head of fish in cm (numerical)
\end{enumerate}

\hypertarget{read-the-data}{%
\subsection{Read the data}\label{read-the-data}}

\begin{Shaded}
\begin{Highlighting}[]
\FunctionTok{setwd}\NormalTok{(}\StringTok{"E:/Data"}\NormalTok{)}
\CommentTok{\# Import library you may need}
\FunctionTok{library}\NormalTok{(car)}
\end{Highlighting}
\end{Shaded}

\begin{verbatim}
## Loading required package: carData
\end{verbatim}

\begin{Shaded}
\begin{Highlighting}[]
\CommentTok{\# Read the data set}
\NormalTok{fishfull }\OtherTok{=} \FunctionTok{read.csv}\NormalTok{(}\StringTok{"Fish.csv"}\NormalTok{,}\AttributeTok{header=}\NormalTok{T, }\AttributeTok{fileEncoding =} \StringTok{\textquotesingle{}UTF{-}8{-}BOM\textquotesingle{}}\NormalTok{)}
\NormalTok{row.cnt }\OtherTok{=} \FunctionTok{nrow}\NormalTok{(fishfull)}
\CommentTok{\# Split the data into training and testing sets}
\NormalTok{fishtest }\OtherTok{=}\NormalTok{ fishfull[(row.cnt}\DecValTok{{-}9}\NormalTok{)}\SpecialCharTok{:}\NormalTok{row.cnt,]}
\NormalTok{fish }\OtherTok{=}\NormalTok{ fishfull[}\DecValTok{1}\SpecialCharTok{:}\NormalTok{(row.cnt}\DecValTok{{-}10}\NormalTok{),]}

\FunctionTok{head}\NormalTok{(fish)}
\end{Highlighting}
\end{Shaded}

\begin{verbatim}
##   Weight Species Body.Height Total.Length Diagonal.Length  Height  Width
## 1    300    Pike        34.8         37.3            39.8  6.2884 4.0198
## 2    242   Bream        23.2         25.4            30.0 11.5200 4.0200
## 3    500   Bream        29.1         31.5            36.4 13.7592 4.3680
## 4    600   Bream        29.4         32.0            37.2 15.4380 5.5800
## 5    345    Pike        36.0         38.5            41.0  6.3960 3.9770
## 6   1000   Perch        40.2         43.5            46.0 12.6040 8.1420
\end{verbatim}

\emph{Please use fish as your data set for the following questions
unless otherwise stated.}

\hypertarget{question-1-exploratory-data-analysis-8-points}{%
\section{Question 1: Exploratory Data Analysis {[}8
points{]}}\label{question-1-exploratory-data-analysis-8-points}}

\textbf{(a) Create a box plot comparing the response variable,
\emph{Weight}, across the multiple \emph{species}. Based on this box
plot, does there appear to be a relationship between the predictor and
the response?}

\begin{Shaded}
\begin{Highlighting}[]
\FunctionTok{attach}\NormalTok{(fish)}
\FunctionTok{boxplot}\NormalTok{(Weight }\SpecialCharTok{\textasciitilde{}}\NormalTok{ Species)}
\end{Highlighting}
\end{Shaded}

\includegraphics{HW3_starter_template-3_files/figure-latex/unnamed-chunk-2-1.pdf}

\textbf{Answer:}

There appears to be a relationship between Species and Weight. In the
chart above, we can see that Parkki, Roach and Smelt have relatively
lower Weight on average, while Bream, Pike and Whitefish have relatively
higher weight on average. However, we need further investigations to
figure out if the this difference in average weight is statistically
significant.

\textbf{\texttt{\textless{}br\ /\textgreater{}}}
\textbf{\texttt{\textless{}br\ /\textgreater{}}}

\textbf{(b) Create scatterplots of the response, \emph{Weight}, against
each quantitative predictor, namely} Body.Height\textbf{,}
Total.Length\textbf{,} Diagonal.Length\textbf{,} Height\textbf{, and}
Width\textbf{. Describe the general trend of each plot. Are there any
potential outliers?}

\textbf{(c) Display the correlations between each of the quantitative
variables. Interpret the correlations in the context of the
relationships of the predictors to the response and in the context of
multicollinearity.}

\textbf{(d) Based on this exploratory analysis, is it reasonable to
assume a multiple linear regression model for the relationship between
\emph{Weight} and the predictor variables?}

\hypertarget{question-2-fitting-the-multiple-linear-regression-model-8-points}{%
\section{Question 2: Fitting the Multiple Linear Regression Model {[}8
points{]}}\label{question-2-fitting-the-multiple-linear-regression-model-8-points}}

\emph{Create the full model without transforming the response variable
or predicting variables using the fish data set. Do not use fishtest}

\textbf{(a) Build a multiple linear regression model, called model1,
using the response and all predictors. Display the summary table of the
model.}

\textbf{(b) Is the overall regression significant at an} \(\alpha\)
level of 0.01? Explain.

\textbf{(c) What is the coefficient estimate for \emph{Body.Height}?
Interpret this coefficient.}

\textbf{(d) What is the coefficient estimate for the \emph{Species}
category Parkki? Interpret this coefficient.}

\hypertarget{question-3-checking-for-outliers-and-multicollinearity-6-points}{%
\section{Question 3: Checking for Outliers and Multicollinearity {[}6
points{]}}\label{question-3-checking-for-outliers-and-multicollinearity-6-points}}

\textbf{(a) Create a plot for the Cook's Distances. Using a threshold
Cook's Distance of 1, identify the row numbers of any outliers.}

\textbf{(b) Remove the outlier(s) from the data set and create a new
model, called model2, using all predictors with \emph{Weight} as the
response. Display the summary of this model.}

\textbf{(c) Display the VIF of each predictor for model2. Using a VIF
threshold of max(10, 1/(1-}\(R^2\)) what conclusions can you draw?

\hypertarget{question-4-checking-model-assumptions-6-points}{%
\section{Question 4: Checking Model Assumptions {[}6
points{]}}\label{question-4-checking-model-assumptions-6-points}}

\emph{Please use the cleaned data set, which have the outlier(s)
removed, and model2 for answering the following questions.}

\textbf{(a) Create scatterplots of the standardized residuals of model2
versus each quantitative predictor. Does the linearity assumption appear
to hold for all predictors?}

\textbf{(b) Create a scatter plot of the standardized residuals of
model2 versus the fitted values of model2. Does the constant variance
assumption appear to hold? Do the errors appear uncorrelated?}

\textbf{(c) Create a histogram and normal QQ plot for the standardized
residuals. What conclusions can you draw from these plots?}

\hypertarget{question-5-partial-f-test-6-points}{%
\section{Question 5: Partial F Test {[}6
points{]}}\label{question-5-partial-f-test-6-points}}

\textbf{(a) Build a third multiple linear regression model using the
cleaned data set without the outlier(s), called model3, using only
\emph{Species} and \emph{Total.Length} as predicting variables and
\emph{Weight} as the response. Display the summary table of the model3.}

\textbf{(b) Conduct a partial F-test comparing model3 with model2. What
can you conclude using an} \(\alpha\) level of 0.01?

\hypertarget{question-6-reduced-model-residual-analysis-and-multicollinearity-test-7-points}{%
\section{Question 6: Reduced Model Residual Analysis and
Multicollinearity Test {[}7
points{]}}\label{question-6-reduced-model-residual-analysis-and-multicollinearity-test-7-points}}

\textbf{(a) Conduct a multicollinearity test on model3. Comment on the
multicollinearity in model3.}

\textbf{(b) Conduct residual analysis for model3 (similar to Q4).
Comment on each assumption and whether they hold.}

\hypertarget{question-7-transformation-9-pts}{%
\section{Question 7: Transformation {[}9
pts{]}}\label{question-7-transformation-9-pts}}

\textbf{(a) Use model3 to find the optimal lambda, rounded to the
nearest 0.5, for a Box-Cox transformation on model3. What
transformation, if any, should be applied according to the lambda value?
Please ensure you use model3}

\textbf{(b) Based on the results in (a), create model4 with the
appropriate transformation. Display the summary.}

\textbf{(c) Perform Residual Analysis on model4. Comment on each
assumption. Was the transformation successful/unsuccessful?}

\hypertarget{question-8-model-comparison-2-pts}{%
\section{Question 8: Model Comparison {[}2
pts{]}}\label{question-8-model-comparison-2-pts}}

\textbf{(a) Using each model summary, compare and discuss the R-squared
and Adjusted R-squared of model2, model3, and model4.}

\hypertarget{question-9-prediction-8-points}{%
\section{Question 9: Prediction {[}8
points{]}}\label{question-9-prediction-8-points}}

\textbf{(a) Predict Weight for the last 10 rows of data (fishtest) using
both model3 and model4. Compare and discuss the mean squared prediction
error (MSPE) of both models.}

\textbf{(b) Suppose you have found a Perch fish with a Body.Height of 28
cm, and a Total.Length of 32 cm. Using model4, predict the weight on
this fish with a 90\% prediction interval. Provide an interpretation of
the prediction interval.}

\end{document}
